\section{Domande frequenti all'esame}

\begin{itemize}
    \item
    Costruire una tavola di routing e descrivere l'algoritmo di routing basato sullo strato di collegamento o sul vettore delle distanze, mostrandone il funzionamento; descrivere inoltre il problema dell'instabilità (relativo a un ciclo di due nodi) e come può essere risolto
    
    \item
    Mostrare in modo dettagliato il processo di inoltro di un datagram
    
    \item
    Frammentare un datagram secondo certi dati forniti dalla traccia
    
    \item
    Descrivere caratteristiche e funzionalità del protocollo TCP
    
    \item
    Descrivere il controllo del flusso nel TCP
    
    \item
    Descrivere il protocollo ARP, il suo funzionamento e il formato dei messaggi scambiati
    
    \item
    Descrivere il protocollo ICMP, in particolare i messaggi di notifica degli errori
    
    \item
    Creare sottoblocchi di una possibile allocazione di un blocco di indirizi ad un gruppo di $n$ persone
    
    \item
    Mostrare i tre segmenti TCP utilizzati nella fase di apertura della connessione nelle ipotesi di certi valori \textit{rwnd}
    
    \item
    Descrivere i concetti fondamentali relativi alla velocità di trasferimento durante la trasmissione dati, i limiti e le differenze fra il Teorema di Nyquist e il Teorema di Shannon; inoltre, data la banda $B$ e l'$SNR$ di un canale, determinare a quale velocità si può spedire e quanti livelli del segnale si possono utilizzare. 
    
    \item
    Descrivere il multiplexing e le tecniche fondamentali per implementarlo
    
    \item
    Descrivere i protocolli per canali senza rumore dello strato di collegamento
    
    \item
    Descrivere il funzionamento del DNS
    
    \item
    Descrivere le tecniche di codifica unipolari e polari utilizzate nella codifica digtale-digitale
    
    \item
    Descrivere le reti di commutazione di circuito
    
    \item
    Descrivere i codici lineari utilizzati per la correzzione degli errori, evidenziando le principali differenza tra codici a parità, codici a parità bidimensionale e codici di Hamming
    
    \item
    Descrivere i protocolli RARP, BOOT e DHCP
    
    \item
    Descrivere il comando traceroute nei sistemi Unix mostrandone il funzionamento su una rete d'esempio
    
    \item
    Sottolineare le differenze fra segnali semplici e composti
    
    \item
    Dato un indirizzo IP, determinare il numero di indirizzi nella rete, nonché il primo e l'ultimo indirizzo del blocco
    
    \item
    Descrivere la procedura di handshake in una connessione TCP
    
    \item
    Descrivere i concetti fondamentali relativi alle codifiche analogico-digitale e digitale-analogico
    
    \item
    Descrivere le differenze sostanziali fra reti a commutazione di circuito e reti a commutazione di pacchetto
    
    \item
    Descrivere il formato di un datagram IP
    
    \item
    Descrivere le misure utilizzate per misurare l'efficienza di una rete
    
    \item
    Descrivere le funzionalità del protocollo IP, descrivendone in dettaglio la frammentazione; descrivere i campi dell'intestazione IPv4
    
    \item
    Si fornisca la notazione decimale e binaria di un certo indirizzo IPv4; trovare l'errore in un indirizzo IPv4
    
    \item
    Descrivere il meccanismo di congestione nel TCP
    
    \item
    Descrivere in dettaglio i campi dell'header di un segmento TCP e spiegare i meccanismi di instaurazione e chiusura delle connessioni TCP
    
    \item
    Descrivere il funzionamento del three-way handshake
    
    \item
    Descrivere il meccanismo di indirizzamento con e senza classi
    
    \item
    Descrivere i protocolli ALOHA e CSMA
    
    \item
    Descrivere il protocollo UDP
    
    \item
    Svolgere un esercizio pratico sulla congestione nel TCP (\textit{slow-start}, \textit{
ssthresh})
\end{itemize}