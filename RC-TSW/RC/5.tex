\section{Conversione dei segnali}

Ricordiamo che una rete di comunicazione è progettata per spedire dati da un punto all'altro. I dati devono essere convertiti in segnali analogici o digitali, e sono poi trasportati dal mezzo trasmissivo.

Se la forma originale dei dati è diversa dal tipo di segnale supportato dal canale trasmissivo, può essere necessaria una conversione "dati-segnali".

\begin{itemize}
    \item Analogico - Digitale
    \item Digitale - Analogico
    \item Analogico - Analogico
    \item Digitale - Digitale
\end{itemize}

Le necessità di convertire i primi due set è ovvia. Tuttavia, anche quando i dati sono nella stessa forma del segnale trasportabile può essere necessaria una conversione. 

I dati digitale possono essere rappresentati in vari modi da un sdegnale digitale (codifica di linea); analogamente, un segnale analogico può dover essere modulato su altre frequenze per essere trasmesso su un canale analogico.

\subsection{Conversione digitale-digitale}

Rappresentazione di dati digitali usando segnali digitali. Si usano le tecniche di codifica di linea e di codifica a blocchi.

\vspace{3mm}

La codifica di linea assume che i dati siano memorizzati nella memoria del computer come sequenze di bit. La codifica in sé permette di convertire una sequenza di bit in segnali digitali. Il mittente esegue la conversione binaria dei dati, e il destinatario ricrea i dati digitali dal segnale digitale.

In questa tecnica risultano necessari un codificatore e un decodificatore. Inoltre, è possibile caratterizzarla come segue:

\begin{itemize}
    \item \textbf{Unità di base dei dati e dei segnali.} Si distinguono gli elementi dei dati, intesi come i bit, e gli elementi del segnale, e si valuta il loro rapporto.
    \item \textbf{Velocità dei dati e velocità del segnale.} La velocità dei dati spedibili in 1s (bps), è indicata con N, e velocità degli elementi del segnale spedibili in 1s (baud), è indicata con S. La formula risultante sarà \(S=c*n*\frac{1}/r baud\).
    \item \textbf{Linea di base.} Nella codifica di un segnale digitale, il destinatario calcola in tempo reale la media della potenza del segnale che riceve (linea di base). Il segnale ricevuto viene confrontato con la linea di base per determinare il valore dell'elemento del segnale  che si deve decodifcare. Lunghe sequene di valori uguali falsano la media.
    \item \textbf{Componenti DC.} Quando il vantaggio di un segnale è costante per un certo periodo di tempo vengono utilizzare frequenze molto basse, vicine allo zero. Risulta un problema, poiché' su alcuni collegamenti non è è possibile il passaggio di frequenze basse.
    \item \textbf{Sincronizzazione automatica.} Per interpretare correttamente la lettura del segnale da parte del destinatario, deve corrispopndere agli intervalli di tempo usati dal mittente per generare il sdegnale. I clock possono non essere perfettamente sincronizzati, quindi serve introdurre degli elementi di sincronizzazione.
\end{itemize}

\subsection{Codifiche di linea}

Gli schemi di codifica di linea sono shcemi che permettono la rappresentazione dei dati (bit) con elementi del segnale digitale. Se ne hanno diversi, fra cui l'\textbf{unipolare} (NRZ), il \textbf{polare} (NRZ, RZ e bifase), il \textbf{bipolare} (AMI e pseudoternaria), il \textbf{multilivello} (2B/1Q, 8B/6T, 4D-PAM5) e il \textbf{multilinea} (MLT-3).

\vspace{3mm}

In uno schema unipolare, i valori di tutti i livelli del segnale che vengono utilizzati hanno lo stesso segno (positivi o negativi), e viene detto \textbf{Non-Return-to-Zero} (NRZ). 1 rappresenta un voltaggio positivo, 0 un voltaggio zero.

\vspace{3mm}

In tutti gli schemi polari, i livelli del segnale possono assumere sia valori positivi che negativi. Nella codifica NRZ polare, vengono utilizzati 2 livelli di voltaggio: positivi e negativi. Nel NRZ-L il livello del voltaggio determina il valore dei bit; nel NRZ-I, il valore dei bit è determinato dall'asdsenza o preseza di un cambio di livello del voltaggio.

\vspace{3mm}

Il problema principale degli schemi polari è, ad esempio, che nella codifica NRZ-L, se c'è una sequenza lunga di 0 o di 1, il valore medio della potenza del segnale si avvicinerà sempre più al valore del voltaggio positivo (o negativo) che rappresenta 0 o 1. Il destinatario potrebbe dunque avere problemi a distinguere una tensione positiva o negativa da un'assenza di tensione. Vi è inoltre il problema della sincronizzazione dei clock fra mittente e destinatario, e dei componmenti DC. Indubbiamente, il problema più accentuato negli schema NRZ è quello della scrfinonizzazione.

\vspace{3mm}

La codifica RZ risolve questo problema. Utilizza 3 livelli di segnale, cioè positivo, negativo e zero, e il valore zero rappresenta una tensione negativo, mentre il valore 1 una tensione positiva. Il segnale torna sempre a zero al centro di ogni bit e rimane su tale valore fino all'inizio del prossimo bit.

\vspace{3mm}

L'idea dela transizione al centro di ogni bit (RZ) combinata con la codifica NRZ-L produce la codifica Manchester. Ogni singolo bit viene rappresentato da due elementi del segnale. La combinazione fra NRZ-I e RZ produce, invece, la codifica Manchester differenziale, dove ogni singolo bit viene rappresentato da due elementi del segnale.

\vspace{3mm}

Nella codifica bifase, la transizione al centro viene usata per la sincronizzazione dei clock. La velocità richiesta è due volte quella richiesta dale codifiche NRZ.

\vspace{3mm}

Nella codifica bipolare, vengono utilizzati 3 livelli del segnale: positivo, negativo e zero. Il valore di un bit viene rappresdentato o da un voltaggio pari a zero o da un voltaggio diverso da zero. Richiede la stessa velocità di segnale, ma non componenti DC. Per lunghe sequenze di 0, il vantaggio è costante. La codifica NRZ concentra la maggior parte dell'energia vicino alla frequenza nulla, e quindi non è adatta a canali che hanno cattive prestazioni per questo tipo di frequenze, mentre gli schemi bipolari concentrano l'energia attorno alla frequenza N/2.

\vspace{3mm}

La codifica \textbf{AMI} (Alternate Mark Inversion) rappresenta con 0 il voltaggio nullo, 1 voltaggi positivi e negativi che si alternano, ed è utilizzata per le comunicazioni a grande distanza. La codifica pseudoternaria utilizza la rappresentazione opposta.

\vspace{3mm}

In ogni caso, si ha la necessità di aumentare la velocità dei dati. Le codifiche multilivello hanno l'obiettivo di incrementare il numero di bit spediti in media per ogni elemento di segnale codificando una sequenza di m bit con una sequenza di n elementi del segnale. Sono solitamente indicate con sigle mBnL, dove m è la lunghezza della sequenza dei dati, B indica che i dati sono binari, n è la lunghezza della sequenza del segnale, e L indica il numero di livelli.

La \textbf{2B1Q} codifica 2 bit con un elemento di segnale a 4 livelli, e cioè ad una velocità doppia rispetto al NRZ-L.

\subsection{Codifiche a blocchi}

Per ottenere la sincronizzazione dei clock, occorre introdurre ridondanza. La codifica a blocchi viene solitamente chiamata codifica mB/nB, perchè trasforma una parola sorgente di m bit in una parola di n bit con n>m. 

Ad esempio, La codifica a blocchi 4B/5B è sincronizzabile e non produce mai una sequenza con più di 3 zero consecutivi.

\vspace{3mm}

In genere, \textbf{vengono usate sia codifiche di linea} (per rappresentare i dati), \textbf{sia cdifiche a blocchi} (per ottenere la sincronizzazione).